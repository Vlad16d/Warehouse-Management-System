\documentclass[a4paper,12pt]{article}

% Ustawienia strony i polskie znaki
\usepackage[utf8]{inputenc}
\usepackage[T1]{fontenc}
\usepackage{polski}
\usepackage[margin=1in]{geometry}
\usepackage{graphicx}

\begin{figure}
    \centering
    \includegraphics[width=1\linewidth]{logoWSIiZ.PNG}
\end{figure}

% Dane autora i dokumentu
\title{Dokumentacja Projektu: System Zarządzania Magazynem}
\author{Uladzislau Bainiarovich \\ Numer albumu: w68339}
\date{2025}

\begin{document}

% Strona tytułowa
\maketitle
\thispagestyle{empty}
\newpage

% Spis treści
\tableofcontents
\newpage

% Rozdział 1: Wstęp
\section{Wstęp}
System Zarządzania Magazynem to aplikacja, która umożliwia użytkownikom zarządzanie produktami i użytkownikami w bazie danych. Projekt został zrealizowany w języku C\# z wykorzystaniem bazy danych Microsoft SQL Server.

% Rozdział 2: Cel projektu
\section{Cel projektu}
Celem projektu było stworzenie funkcjonalnego systemu, który umożliwia:
\begin{itemize}
    \item Dodawanie, usuwanie i edytowanie użytkowników oraz produktów.
    \item Wyświetlanie listy użytkowników i produktów.
    \item Zarządzanie bazą danych w sposób efektywny i bezpieczny.
\end{itemize}

% Rozdział 3: Opis funkcjonalności
\section{Opis funkcjonalności}
System oferuje następujące funkcjonalności:
\begin{itemize}
    \item Operacje CRUD (Create, Read, Update, Delete) dla użytkowników i produktów.
    \item Walidację danych wprowadzanych przez użytkownika.
    \item Obsługę wyjątków i błędów.
    \item Integrację z bazą danych Microsoft SQL Server.
\end{itemize}

% Rozdział 4: Architektura projektu
\section{Architektura projektu}
Projekt wykorzystuje wzorzec warstwowy, który składa się z następujących komponentów:
\begin{itemize}
    \item \textbf{DatabaseManager}: Klasa odpowiedzialna za komunikację z bazą danych.
    \item \textbf{PenaltyCalculator}: Klasa odpowiedzialna za obliczanie kar
    \item \textbf{FileManager}: Klasa umożliwiająca zarządzanie danymi w systemie plików
    \item \textbf{Program}: Główna klasa programu odpowiedzialna za interfejs użytkownika
    \item \textbf{StatisticsManager}: Klasa zarządzająca generowaniem statystyk w systemie. 
    \item \textbf{Validator}: Klasa odpowiedzialna za walidację danych wprowadzanych przez użytkownika. 
    \item \textbf{Warehouse}: Klasa zarządzająca użytkownikami i produktami.
    \item \textbf{User}: Reprezentacja użytkownika w systemie.
    \item \textbf{Product}: Reprezentacja produktu w systemie.
\end{itemize}

Struktura bazy danych składa się z dwóch głównych tabel:
\begin{itemize}
    \item \textbf{Users}: Przechowuje dane użytkowników (Id, Name).
    \item \textbf{Products}: Przechowuje dane produktów (Id, Name, Quantity, ExpiryDate, UserId).
\end{itemize}

% Rozdział 5: Instrukcja użytkownika
\section{Instrukcja użytkownika}
Aby uruchomić system zarządzania magazynem, wykonaj następujące kroki:
\begin{enumerate}
    \item Skonfiguruj bazę danych Microsoft SQL Server i zainicjalizuj ją przy użyciu dostarczonego skryptu SQL.
    \item Uruchom aplikację z poziomu Visual Studio.
    \item Korzystaj z menu aplikacji, aby dodawać użytkowników i produkty, a także przeglądać dane.
\end{enumerate}

% Rozdział 6: Podsumowanie
\section{Podsumowanie}
Projekt "System Zarządzania Magazynem" jest przykładem aplikacji opartej na języku C\#, która demonstruje podstawowe operacje na bazach danych oraz obsługę błędów i walidację danych. System został zaprojektowany w sposób modularny, co ułatwia jego rozbudowę w przyszłości.

\end{document}
